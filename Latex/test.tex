\documentclass[letterpaper,12pt]{article}
\usepackage{listings}

\lstset{language=Python, 
    basicstyle=\ttfamily,
    keywordstyle=\color{blue}\ttfamily,
    stringstyle=\color{red}\ttfamily,
    commentstyle=\color{green}\ttfamily,
    morecomment=[l][\color{magenta}]{\#}
}

\usepackage{tabularx} % extra features for tabular environment
\usepackage{amsmath}  % improve math presentation
\usepackage{graphicx} % takes care of graphic including machinery
\usepackage[margin=1in,letterpaper]{geometry} % decreases margins
\usepackage{cite} % takes care of citations
\usepackage[final]{hyperref} % adds hyper links inside the generated pdf file
\hypersetup{
    colorlinks=true,       % false: boxed links; true: colored links
    linkcolor=blue,        % color of internal links
    citecolor=blue,        % color of links to bibliography
    filecolor=magenta,     % color of file links
    urlcolor=blue         
}
\usepackage{blindtext}
%++++++++++++++++++++++++++++++++++++++++


\begin{document}

\title{Solución de Sistemas de ecuaciones lineales y factorización LU con Python}
\author{Johan Posada y Juan Morales}
\date{Mayo 14 2024}
\maketitle

\section{Introducción}
... El objetivo de este documento es explicar cómo se puede construir un programa en Python que permita resolver sistemas de ecuaciones lineales de única solución.
\section{Construyendo el programa}
En un principio el programa se había construido sin usar POO, sin embargo se decidió hacer uso de este paradigma de la programación para
mejorar su estructura. El código se puede acceder en el perfil de GitHub del desarrollador del código: \url{https://github.com/johanP051/Equations-systems.git}

Para empezar, se debe solicitar al usuario que ingrese el número de ecuaciones y el número de variables 
del sistema que siga la siguiente forma:

\begin{align*}
    a_{11}x_1 + a_{12}x_2 + \cdots + a_{1n}x_n = b_1, \quad
    \\
    a_{21}x_1 + a_{22}x_2 + \cdots + a_{2n}x_n = b_2, \quad
    \\
    a_{n1}x_1 + a_{n2}x_2 + \cdots + a_{nn}x_n = b_n
    \end{align*}


\[
\left(
\begin{array}{ccc|c}
a_{11} & a_{12} & \cdots & b_1 \\
a_{21} & a_{22} & \cdots & b_2 \\
\vdots & \vdots & \ddots & \vdots \\
a_{n1} & a_{n2} & \cdots & b_n \\
\end{array}
\right)
\]

Donde la última columna es el el vector de igualdad.


\begin{thebibliography}{99}
\bibitem{torres_solis}
Torres Solís, M., Villalobos Castillo, N. (s.f). Factorización LU. Universidad del Bío-Bío. Recuperado de \url{http://repobib.ubiobio.cl/jspui/bitstream/123456789/1811/1/Torres_Solis_Marcos.pdf}
\end{thebibliography}
\end{document}